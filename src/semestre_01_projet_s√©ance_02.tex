\documentclass[10pt]{beamer}
\usetheme{metropolis}

\usepackage{amsmath, booktabs, fontawesome5, natbib, subfigure, xcolor}

\usepackage[font=small,skip=0pt]{caption}

\usepackage{pgfplots}
\usepgfplotslibrary{dateplot}

\usepackage{xspace}
\newcommand{\themename}{\textbf{\textsc{metropolis}}\xspace}

\usepackage{graphicx}
\graphicspath{{../imgs/}}

\usepackage[T1]{fontenc}
\usepackage[french]{babel}

\usepackage{appendixnumberbeamer}

\usepackage{tikz}
\usetikzlibrary{shapes.geometric, arrows}
\tikzstyle{rect} = [rectangle, rounded corners, minimum width=2cm, minimum height=1.5cm, text centered, draw=black, fill=black!30]
\tikzstyle{squa} = [square, rounded corners, minimum width=1.25cm, minimum height=1.25cm, text centered, draw=black, fill=black!30]
\tikzstyle{elli} = [ellipse, minimum width=1.25cm, minimum height=1cm, text centered, draw=black, fill=black!30]
\tikzstyle{circ} = [circle, minimum width=1cm, minimum height=1cm, text centered, draw=black, fill=black!30]
\tikzstyle{arrow} = [thick,->,>=stealth]
\tikzstyle{drrow} = [thick,<->,>=stealth]
\tikzstyle{dline} = [dashed, ->, >=stealth]
\tikzstyle{dotted} = [densely dotted, ->, >=stealth]

\def\firstcircle{(90:1.75cm) circle (2.5cm)}
\def\secondcircle{(210:1.75cm) circle (2.5cm)}
\def\thirdcircle{(330:1.75cm) circle (2.5cm)}
\tikzset{fontscale/.style = {font=\relsize{#1}}}

\titlegraphic{\hfill\includegraphics[height=1.25cm]{~/Documents/scpobx/logo.pdf}}
\title{Méthodes des sciences sociales}
\subtitle{Séance 2: L'objet de recherche}
\author{Mickael Temporão}
% \institute{Sciences Po Bordeaux}
\date{}

\begin{document}

\maketitle

\begin{frame}{Méthodes des sciences sociales}
    \onslide<1->{
        \begin{block}{Ordre du jour}
            \begin{itemize}
                \item<2-> Discussion : \textit{KKV} Chapitre 1
                \item<3-> Composantes d'un projet de recherche
% TODO: Plan du travail à remettre
                \item<4-> Objet, sources et problématique
                    % Caractéristiques d'une bonne questions
                    % Identifier un objet, sources et problématiques
            \end{itemize}
        \end{block}
    }
    \vspace{24pt}
    \onslide<6->{
        \begin{block}{\faTrophy~BONUS}
        \begin{itemize}
            \item[] \faGithubSquare~Github
            \item[] \faRProject~R
            \item[] \faPollH~Rétrospective
        \end{itemize}
        \end{block}
    }
\end{frame}

\begin{frame}{Fonctionnement du cours}

    \onslide<2->{
    \begin{block}{Déroulement d'une séance}
    \begin{center}
        \begin{tikzpicture}[node distance=1.5cm]
        \onslide<3->{
            \node (pres) [rect, xshift=-8cm] {Présentation};
        }
        \onslide<4->{
            \node (acti) [rect, xshift=2cm, right of=pres] {Activité(s)};
            \draw [arrow] (pres) -- (acti);
        }
        \onslide<5->{
            \node (disc) [rect, xshift=2cm, right of=acti] {Discussion};
            \draw [arrow] (acti) -- (disc);
        }
        \end{tikzpicture}
    \end{center}
    \end{block}
    }


\end{frame}

\section{Bonus}

\begin{frame}{Bonus...}
        \begin{itemize}
            \item[] \faSlack ~Slack
            \item[] \faGithubSquare~Github
        \end{itemize}
\end{frame}


\begin{frame}[standout]
    \begin{flushright}
        \vspace{100pt}
        \small\faTwitter~$@$mickaeltemporao
    \end{flushright}
\end{frame}

\end{document}
